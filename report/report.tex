% !TEX TS-program = pdflatex
% !TEX encoding = UTF-8 Unicode

% This is a simple template for a LaTeX document using the "article" class.
% See "book", "report", "letter" for other types of document.

\documentclass[11pt]{article} % use larger type; default would be 10pt

\usepackage[utf8]{inputenc} % set input encoding (not needed with XeLaTeX)

%%% Examples of Article customizations
% These packages are optional, depending whether you want the features they provide.
% See the LaTeX Companion or other references for full information.

%%% PAGE DIMENSIONS
\usepackage{geometry} % to change the page dimensions
\geometry{a4paper} % or letterpaper (US) or a5paper or....
% \geometry{margin=2in} % for example, change the margins to 2 inches all round
% \geometry{landscape} % set up the page for landscape
%   read geometry.pdf for detailed page layout information

\usepackage{graphicx} % support the \includegraphics command and options

% \usepackage[parfill]{parskip} % Activate to begin paragraphs with an empty line rather than an indent

%%% PACKAGES
\usepackage{booktabs} % for much better looking tables
\usepackage{array} % for better arrays (eg matrices) in maths
\usepackage{paralist} % very flexible & customisable lists (eg. enumerate/itemize, etc.)
\usepackage{verbatim} % adds environment for commenting out blocks of text & for better verbatim
\usepackage{subfig} % make it possible to include more than one captioned figure/table in a single float
% These packages are all incorporated in the memoir class to one degree or another...

%%% HEADERS & FOOTERS
\usepackage{fancyhdr} % This should be set AFTER setting up the page geometry
\pagestyle{fancy} % options: empty , plain , fancy
\renewcommand{\headrulewidth}{0pt} % customise the layout...
\lhead{}\chead{}\rhead{}
\lfoot{}\cfoot{\thepage}\rfoot{}

%%% SECTION TITLE APPEARANCE
\usepackage{sectsty}
\allsectionsfont{\sffamily\mdseries\upshape} % (See the fntguide.pdf for font help)
% (This matches ConTeXt defaults)

%%% ToC (table of contents) APPEARANCE
\usepackage[nottoc,notlof,notlot]{tocbibind} % Put the bibliography in the ToC
\usepackage[titles,subfigure]{tocloft} % Alter the style of the Table of Contents
\renewcommand{\cftsecfont}{\rmfamily\mdseries\upshape}
\renewcommand{\cftsecpagefont}{\rmfamily\mdseries\upshape} % No bold!

\usepackage{hyperref}
\usepackage{amsmath}


%%% END Article customizations

%%% The "real" document content comes below...

\title{FIXME: SeRPE}
\author{Yehuda Katz, Phil Nguyen, Peratham Wiriyathammabhum}
%\date{} % Activate to display a given date or no date (if empty),
         % otherwise the current date is printed 

\begin{document}
\maketitle

\begin{abstract}
In this project, we implement a Sequential Refinement Planning Engine
(SeRPE)~\cite{serpe-draft} for Pyhop~\cite{pyhop},
a simple planner written in Python.
\end{abstract}

\section{Introduction}

%% HTN
\textbf{Hierarchical Task Network (HTN) Planning} is an automated planning method that uses networks
to describe dependencies among actions.
For a particular planning problem, HTN planning takes each task and attempts to find a solution
by decomposing the problem into subtasks. The decomposition stops when all the tasks are broken down to \textit{primitive tasks} which can be directly performed by \textit{planning operators}.
A task network also contains 
\textit{compound tasks}, each of which is a sequence of actions,
\textit{and goal tasks}, each of which satisfies a condition.

%% TODO what is Pyhop
\textbf{Pyhop} is a simple implementation of HTN written in Python. 
Pyhop represents states of the world using Python variable binding
 instead of logical propositions. Operators and methods in Pyhop are
 just Python functions with the current state as an argument.

%% TODO what is SePRE
\textbf{Sequential Refinement Planning (SeRPE)} is an improvement to Refinement Planning. Refinement Planning algorithms may need to make choices to determine what their next course of action will be. At each choice, SeRPE looks at all the available choices and using descriptive models, attempts to predict the outcome of each possible action.
Instead of considering each action having a particular effect (like HTN), a refinement planning will have a body which
 is a program. The program can have assignments, actions, tasks or
 even some control structures like if-then-else or while-loop.

SeRPE assumes all actions to be deterministic and all methods are 
sequential and cannot be executed in parallel.

\section{Related work}
There is a long history of planning software for a variety of different algorithms. One of the first HTN planners was published in 1975.
\begin{itemize}
  \item Pyhop. HTN Planner on which our SeRPE implementation is based. \cite{pyhop}
  
\end{itemize}

\section{Approach}
We made an attempt to extend Pyhop to behave like SeRPE. 
First, we modified the use of HTN by replacing subtasks with 
body. Body can be any kind of program which consists of
 Python statements like if-else or while-loop.
Second, we use Python's `eval' function to implement the 
backtracking in execution stack for SeRPE. This will be a 
description action model for SeRPE rather than just execute
 the action in RAE. The execution stack will be similar to what
 already in Pyhop except that it will be used in the method's body.

\section{Implementation}

Our implementation is available online at
\begin{center}
\href{https://github.com/yakatz/CMSC722-KNW/tree/master/src}
{\tt https://github.com/yakatz/CMSC722-KNW/tree/master/src}
\end{center}

For an example, we modified the simple\_travel\_example given in Pyhop.
We made a situation where travel\_by\_taxi can cause failure because
the money is not enough. So, we add a refinement action 'go\_to\_bank'
which will help increase the money and execute the 'travel' task again.

%% TODO code screenshots

\section{Conclusion}

We have attempted to implement an Pyhop extension for SePRE.
By using Python's 'eval' function, we can simulate the description 
action model for SeRPE.

\bibliographystyle{plain}
\bibliography{report}

\end{document}
